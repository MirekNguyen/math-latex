\documentclass{article}
\usepackage{amsmath}
\usepackage{enumitem}
\usepackage{pgfplots}
\usepackage{hyperref}

\begin{document}

\title{Calculus}
\author{Mirek Nguyen}
\date{}

\maketitle
\tableofcontents

\clearpage
\section{Limita}
\subsection{Limita s nekonečnem v podílu}
\begin{itemize}
  \item platí pro $\pm\infty$
\end{itemize}
\begin{align*}
  \lim_{x \to \infty}f(\frac{x^2-1}{x^3+1})&=\frac{mensi}{vetsi}=0 \\
  \lim_{x \to \infty}f(\frac{-x^4+x}{4+x-2x^4})&=\frac{stejny}{stejny}=\frac{-1}{-2}=\frac{1}{2} \\
  \lim_{x \to \infty}f(\frac{x^{11}-x^5}{1-x^{11}})&=\frac{stejny}{stejny}=\frac{1}{-1}=-1 \\
  \lim_{x \to \infty}f(\frac{2x^4-3x+5}{1-x^3})&=\frac{vetsi}{mensi}=\frac{2*\infty}{-1}=-\infty
\end{align*}


\subsection{Asymptoty racionálních funkcí}
\begin{itemize}
  \item počítají se v krajních bodech $D(f)$
\end{itemize}
\begin{enumerate}
  \item Definiční obor
  \item Limita dané funkce (pomocí nul. bodu)
  \item Do jakého $\infty$ se blíží
    \begin{itemize}
      \item $zprava^+$ nebo $zleva^-$ (vybrat si)
      \item je to důkaz, že je asymptotou
    \end{itemize}
  \item Vypočítat šikmou asymptotu typu $y=kx+q$
\end{enumerate}

\clearpage
\section{Derivace}
\subsection{Derivace složené funkce}
$$f(g(x))' = f'(g(x))*g'(x)$$
\begin{align*}
\sqrt{6x+7}' &= f'(g(x))*g'(x)\\
             &= \sqrt{g(x)}'*g'(x)\\
             &= \frac{1}{2*\sqrt{g(x)}}*g'(x)\\
             &= \frac{1}{2*\sqrt{6x+7}}*(6x+7)'\\
             &= \frac{3}{\sqrt{6x+7}}
\end{align*}
\subsection{Tečna a normála}
\begin{enumerate}
  \item Dopočítat souřadnici pro tečný bod
  \item Derivace směrnice tečny a normály
    \begin{itemize}
      \item zderivuju celou (zadanou) rovnici
    \end{itemize}
  \item Dosadit směrnici do rovnice
  \item Převést do tvaru rovnice
\end{enumerate}
\begin{eqnarray*}
y = mx+b & & \text{m je směrnice} \\
t: y-y_t=k_n*(x-x_t) & & \text{rovnice tečny} \\
n: y-y_t=k_t*(x-x_t) & & \text{rovnice normály} \\
k_t = f'(x) & & \text{tečna} \\
k_n = -\frac{1}{f'(x)} & & \text{normála}
\end{eqnarray*}

\subsection{Monotonie}
\begin{enumerate}
  \item Definiční obor
  \item Derivace
  \item Nulové body - znaménko $^+_-$
  \item Intervaly, uzavřenost nul. bodů
    \begin{itemize}
      \item rostoucí
      \item klesající
    \end{itemize}
\end{enumerate}

\subsection{Lokální extrémy}
\begin{enumerate}
  \item Definiční obor
  \item Derivace
  \item Nulové body
    \begin{enumerate}[label=(\alph*)]
      \item dosadit do derivace
      \item znaménko
    \end{enumerate}
  \item pouze v nul. bodech jsou extrémy
    \begin{itemize}
      \item může jich být více
      \item ostré lokální maximu, minimum
    \end{itemize}
\end{enumerate}

\subsection{Globální (absolutní) extrémy}
\begin{enumerate}
  \item Definiční obor (může být zadán na intervalu)
  \item Derivace
  \item Nulové body derivace $f'(x)=0$
    \begin{enumerate}[label=(\alph*)]
      \item vypočítat
      \item vyjde konkrétní výsledek
      \item musí být v interavalu D(f)
    \end{enumerate}
  \item K nul. bodům D(f) přidáme hodnotu z $f'(x)=0$
  \item Do funkce f(x) zadáváme hodnoty x z nul. bodů
\end{enumerate}

\subsection{Konvexita, konkávita}
\begin{enumerate}
  \item Definiční obor
  \item 1. derivace a 2. derivace
  \item Nulové body
    \begin{itemize}
      \item podezřelé z inflexe (mění se zde znaménko)
      \item zkontrolovat, zda leží v D(f)
    \end{itemize}
  \item Znaménko nulových bodů
    \begin{itemize}
      \item $\cup$ konvexní (+)
      \item $\cap$ konkávní (-)
    \end{itemize}
  \item Interval konvexity, konkávity
  \item Inflexe - inflexní body
    \begin{itemize}
      \item změna konvexity, konkávity
      \item definovaná, spojitá v bodě
      \item $I_1[x_1;y_1]$
    \end{itemize}
\end{enumerate}

\clearpage
\section{Integrace}
\subsection{Určitý integrál pomocí přímé metody}
\begin{equation}
\int_{a}^{b}f(x)\,dx = \Bigr[ F'(x) \Bigr]_a^b=F(b)-F(a)
\end{equation}

\begin{enumerate}
  \item Převést na primitivní funkci
    \begin{itemize}
      \item rozdělit zlomek na dva (o stejném jmenovateli)
    \end{itemize}
  \item Integrace (abych se zbavil $F'(x)\rightarrow$ negace)
  \item Dosadit $\rightarrow$ budu mít 2 funkce
  \item Odečíst
\end{enumerate}

\subsection{Určitý integrál pomocí substituce}
\begin{itemize}\item pro složené funkce\end{itemize}
\begin{align*}
  \int_a^bf(g(x))*g'(x)\,dx&=
  \Biggr|
    \begin{alignedat}{2}
      g(x) &=t \quad &a \rightarrow g(a) \\
      g'(x)\,dx &=\,dt \quad &b \rightarrow g(b) \\
    \end{alignedat}
  \Biggr| = \\
  \text{I. způsob} &= \int_{g(a)}^{g(b)}f(t)\,dt = \Bigr[F(t)\Bigr]_{g(a)}^{g(b)} = F(g(b))-F(g(a)) \\
  \text{II. způsob} &= \int_?^?f(t)\,dt=\Bigr[F(t)\Bigr]_?^? = \\
  &=\Bigr[F(t)\Bigr]_a^b = F(b)-F(a) \\
  & \text{zde existuje mez, vrátím substituci}
\end{align*}

\subsection{Diferenciální rovnice}
\begin{enumerate}
  \item Převést na formu y=...
  \item Přepsat y $\rightarrow\frac{dy}{dx}$
  \item Vynásobím L a P rovnici "dx"
\end{enumerate}

\begin{align*}
  y'=f(x)*g(x)=\frac{dy}{dx}\\
  y'=\frac{dy}{dx}\\
\end{align*}

\end{document}


