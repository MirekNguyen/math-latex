\documentclass{article}
\usepackage{amsmath}
\usepackage{enumitem}

\begin{document}

\title{Calculus}
\author{Mirek Nguyen}

\maketitle

\section{Limity}
\subsection{Limita s nekonečnem v podílu}
platí pro $\pm\infty$ \newline
\begin{equation}
  \lim_{x \to \infty}f(\frac{x^2-1}{x^3+1})=\frac{mensi}{vetsi}=0
\end{equation}
\begin{equation}
  \lim_{x \to \infty}f(\frac{-x^4+x}{4+x-2x^4})=\frac{-1}{-2}=\frac{1}{2}
\end{equation}
\begin{equation}
  \lim_{x \to \infty}f(\frac{x^{11}-x^5}{1-x^{11}})=\frac{1}{-1}=-1
\end{equation}
\begin{equation}
  \lim_{x \to \infty}f(\frac{2x^4-3x+5}{1-x^3})=\frac{2*\infty}{-1}=-\infty
\end{equation}


\subsection{Asymptoty racionálních funkcí}
\begin{itemize}
  \item počítají se v krajních bodech $D(f)$
\end{itemize}
\begin{enumerate}
  \item Definiční obor
  \item Limita dané funkce (pomocí nul. bodu)
  \item Do jakého $\infty$ se blíží
    \begin{itemize}
      \item $zprava^+$ nebo $zleva^-$ (vybrat si)
      \item je to důkaz, že je asymptotou
    \end{itemize}
  \item Vypočítat šikmou asymptotu typu $y=kx+q$
\end{enumerate}

\section{Derivace}
\subsection{Derivace složené funkce}
$$f(g(x))' = f'(g(x))*g'(x)$$
\begin{align*}
\sqrt{6x+7}' &= f'(g(x))*g'(x)\\
             &= \sqrt{g(x)}'*g'(x)\\
             &= \frac{1}{2*\sqrt{g(x)}}*g'(x)\\
             &= \frac{1}{2*\sqrt{6x+7}}*(6x+7)'\\
             &= \frac{3}{\sqrt{6x+7}}
\end{align*}
\subsection{Tečna a normála}
\begin{enumerate}
  \item Dopočítat souřadnici pro tečný bod
  \item Derivace směrnice tečny a normály
    \begin{itemize}
      \item zderivuju celou (zadanou) rovnici
    \end{itemize}
  \item Dosadit směrnici do rovnice
  \item Převést do tvaru rovnice
\end{enumerate}
\begin{eqnarray*}
y = mx+b & & \text{m je směrnice} \\
t: y-y_t=k_n*(x-x_t) & & \text{rovnice tečny} \\
n: y-y_t=k_t*(x-x_t) & & \text{rovnice normály} \\
k_t = f'(x) & & \text{tečna} \\
k_n = -\frac{1}{f'(x)} & & \text{normála}
\end{eqnarray*}

\subsection{Monotonie}
\begin{enumerate}
  \item Definiční obor
  \item Derivace
  \item Nulové body - znaménko $^+_-$
  \item Intervaly, uzavřenost nul. bodů
    \begin{itemize}
      \item rostoucí
      \item klesající
    \end{itemize}
\end{enumerate}

\subsection{Lokální extrémy}
\begin{enumerate}
  \item Definiční obor
  \item Derivace
  \item Nulové body
    \begin{enumerate}[label=(\alph*)]
      \item dosadit do derivace
      \item znaménko
    \end{enumerate}
  \item pouze v nul. bodech jsou extrémy
    \begin{itemize}
      \item může jich být více
      \item ostré lokální maximu, minimum
    \end{itemize}
\end{enumerate}

\subsection{Globální (absolutní) extrémy}
\begin{enumerate}
  \item Definiční obor (může být zadán na intervalu)
  \item Derivace
  \item Nulové body derivace $f'(x)=0$
    \begin{enumerate}[label=(\alph*)]
      \item
      \item
    \end{enumerate}
  \item 
  \item
  \item
  \item
\end{enumerate}

\end{document}


