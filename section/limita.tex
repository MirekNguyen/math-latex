\section{Limita}
\subsection{Limita s nekonečnem v podílu}
\begin{itemize}
  \item platí pro $\pm\infty$
\end{itemize}
\begin{align*}
  \lim_{x \to \infty}f(\frac{x^2-1}{x^3+1})&=\frac{mensi}{vetsi}=0 \\
  \lim_{x \to \infty}f(\frac{-x^4+x}{4+x-2x^4})&=\frac{stejny}{stejny}=\frac{-1}{-2}=\frac{1}{2} \\
  \lim_{x \to \infty}f(\frac{x^{11}-x^5}{1-x^{11}})&=\frac{stejny}{stejny}=\frac{1}{-1}=-1 \\
  \lim_{x \to \infty}f(\frac{2x^4-3x+5}{1-x^3})&=\frac{vetsi}{mensi}=\frac{2*\infty}{-1}=-\infty
\end{align*}


\subsection{Asymptoty racionálních funkcí}
\begin{itemize}
  \item počítají se v krajních bodech $D(f)$
\end{itemize}
\begin{enumerate}
  \item Definiční obor
  \item Limita dané funkce (pomocí nul. bodu)
  \item Do jakého $\infty$ se blíží
    \begin{itemize}
      \item $zprava^+$ nebo $zleva^-$ (vybrat si)
      \item je to důkaz, že je asymptotou
    \end{itemize}
  \item Vypočítat šikmou asymptotu typu $y=kx+q$
\end{enumerate}

\subsection{L'Hospitalovo pravidlo}
\begin{center}
  Pravidlo: $x_0\in\mathbb{R}\cup\{\pm\infty\}$
  \begin{align*}
    &\lim_{x \to x_0}f(x)=\lim_{x \to x_0}g(x)=0=||\frac{0}{0}|| \\
    &\text{nebo}\\
    &\lim_{x \to x_0}|g(x)|=+\infty=\text{jmenovatel je $\infty$}
  \end{align*}
  \begin{enumerate}
    \item Dosadím $x_0$ do rovnice
    \item Vyjde mi $\frac{0}{0}\rightarrow$ zderivuju a dosadím hodnoty
  \end{enumerate}
  \begin{eqnarray*}
    &\lim_{x \to 0^+}\frac{x^2+2x+4}{x-1}=\frac{4}{0^+}=+\infty\\
    &\lim_{x \to 2}\frac{3x+1}{2-x}
  \end{eqnarray*}
\end{center}

